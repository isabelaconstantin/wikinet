\documentclass[conference]{IEEEtran}
\IEEEoverridecommandlockouts
% The preceding line is only needed to identify funding in the first footnote. If that is unneeded, please comment it out.
\usepackage{cite}
\usepackage{amsmath,amssymb,amsfonts}
\usepackage{algorithmic}
\usepackage{graphicx}
\usepackage{textcomp}
\usepackage{xcolor}
\usepackage{hyperref}
\def\BibTeX{{\rm B\kern-.05em{\sc i\kern-.025em b}\kern-.08em
    T\kern-.1667em\lower.7ex\hbox{E}\kern-.125emX}}
\begin{document}

\title{Hello I'm a title\\
{\footnotesize Network Tour of Data Science Report, January 2019}
}

\author{Isabela Constantin, Ad\'elie Garin, Celia Hacker, Michael Spieler}

\maketitle

\section{Introduction}
Using the Wikipedia data that is now open source, we would like to understand how the number of views on some pages depend on a type of event. We will consider the two following types of events for our analysis:
\begin{enumerate}
\item Expected events such as elections, concerts, political events
\item Unexpected events which could be for example, death of someone famous,a coup d’état, or a natural disaster
\end{enumerate} 


%brexit, 
%https://en.wikipedia.org/wiki/Stan_Lee, https://en.wikipedia.org/wiki/Fukushima_Daiichi_nuclear_disaster, https://en.wikipedia.org/wiki/Chinese_Lunar_Exploration_Program

Choosing several specific events of each type, we will consider the graph built with pages concerning the event as nodes and a two pages are linked if there is a web link from one to the other. We will use both the directed graph, in which one page which is linked to the other leads to only one direction for the concerned edge, and the undirected graph, for which there exists an edge if and only if there is one page linked to the other. 
\medskip

The questions we ask ourselves are the following:  Does the number of views on the pages propagate in the pages in different ways depending on the type of event? If yes, how do they behave?
\medskip

We will consider the quotient of the number of views of the day of the event by the number of views on the day before the event. 
\medskip

To analyse our data and try to answer our questions, we will first start by acquiring the data and building several graphs for each type of event.  Then we explore them and try to get general properties of the graphs, building a general pipeline for our analysis. We then exploit the data, trying to test our hypothesis and finally draw conclusions. 

\section{Data Acquisition}

We constructed the graphs by selecting the Wikipedia article corresponding to each event. We then grew the graphs around them by selecting the pages the event is linking to, and the pages linked from those. To reduce the amount of data we had, we randomly subsampled by giving a higher probability of staying in the graph to the nodes that are higher in the page, meaning they are stated early so they are more likely to be important links. The resulting graphs are directed, unweighted and connected. In addition, we found the number of views per page, that we computed for the day before the event and the day of the event. 
For each page will consider the quotient \[\frac{\text{number of views the day of the event} }{\text{number of views the day before}}.\]  Taking this quotient is a sort of normalization of the number of views. The higher this value is, the more the number of views on the day of the event is big compared to the number of views of the day before. We hence have a value assigned to each node of our graphs. 


\section{Data Exploration}


The six graphs we build are based on the following six events: 

% Other possible Expected events: US elections, release date of a movie maybe, 

\begin{table}[htbp]
\caption{Our graphs}
\begin{center}
\begin{tabular}{|c|c|c|c|}
\hline
\textbf{ } & \textbf{\textit{Event}}& \textbf{\textit{type of event}}& \textbf{\textit{Wikipedia article}} \\
\hline
Graph 1 & Brexit  &Expected & \href{https://en.wikipedia.org/wiki/Brexit}{Brexit}\\
\hline
Graph 2&   &Expected & \\
\hline
Graph 3&  & Expected& \\
\hline
Graph 4 & Death of Stan Lee & Unexpected & \href{ https://en.wikipedia.org/wiki/Stan_Lee }{Stan Lee} \\
\hline
Graph 5& Fukushima nuclear disaster& Unexpected & \href{https://en.wikipedia.org/wiki/Fukushima_Daiichi_nuclear_disaster }{Fukushima}\\
\hline
Graph 6 & Chinese Lunar Exploration  Program & Unexpected &  \href{https://en.wikipedia.org/wiki/Chinese_Lunar_Exploration_Program }{CLEP}\\
\hline
\end{tabular}
\end{center}
\end{table}

\subsection{Basic properties}
We start by analysing some basic properties of each graph, such as the number of nodes, the number of edges, the diameter of the graph, and the average clustering coefficient, with the tools available in Networkx. The results are stated in the  following table. 

\begin{table}[htbp]
\caption{Basic properties of the graphs}
\begin{center}
\begin{tabular}{|c|c|c|c|c|}
\hline
\textbf{Graphs}&\multicolumn{4}{|c|}{\textbf{Properties}} \\
\cline{2-5} 
\textbf{ } & \textbf{\textit{Number of nodes}}& \textbf{\textit{Number of edges}}& \textbf{\textit{Diameter}} & \textbf{\textit{ACC}}\\
\hline
Graph 1 & & & & \\
\hline
Graph 2& & & & \\
\hline
Graph 3& & & & \\
\hline
Graph 4& & & & \\
\hline
Graph 5& & & & \\
\hline
Graph 6 & & & & \\
\hline
\multicolumn{4}{l}{ACC stands for Average Clustering Coefficient.} \\
\end{tabular}
\end{center}
\end{table}

Some other properties we would like to  study are the the degree distributions, the strongly connected components of the graphs and do spectral clustering on the undirected versions of our graphs. 
%subsection for each of those
\subsection{Degree Distribution}
We now plot the degree distribution of each graph: 
%Plots of the degree distributions
\subsection{Strongly Connected Components}
Our graphs are clearly connected in the undirected sense by the way we have constructed them. In this part we compute the number of strongly connected components. A directed graph is said to be strongly connected if there is a directed path connecting any two of its vertices. 

\subsection{Spectral Clustering }
% Spectral clustering will allow us to see if there are hubs in our graphs. We can use to compare how the signal propagates 


\section{Data Exploitation}

We now come to the most important part of our analysis, which is the following: we consider the number of views as a signal on the graph. We would like to see how it behaves. In order to make a sensible comparision of number of views and how these number of views evolve for each graph we use the quotient of the number of views before and at the day of the event. 
The general pipeline of our analysis is :  
%Describe pipeline

\begin{enumerate}
\item 
\item 
\end{enumerate}

%to do: distribution of labels per node and compare to random signal, dirac signal and find one that slowly increases/dicreases 

\section{Conclusion}

Note that if we had the computational power to build a graph with all the pages involved by all the events that we considered together, it would have been much more interesting to see how the number of views evolves locally on the graph. By doing the methods presented in this report, we can only do a “zoom-in” on a specific part of this big graph and analyse it independently of the rest. 


\section*{References}


\cite{laplacian}  
\cite{signalprocessing}
\cite{clustering}

\bibliography{biblio}
\bibliographystyle{plain}

\end{document}
